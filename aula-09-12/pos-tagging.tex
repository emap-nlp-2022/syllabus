\documentclass{beamer}

\usepackage[english]{babel}
% \usepackage[latin1]{inputenc}

\usepackage{times}
\usepackage[T1]{fontenc}

\usepackage{latexsym}
\usepackage{eepic}
\usepackage{url}
\usepackage{graphicx}
\usepackage{epstopdf}

\usepackage{amsmath}
\usepackage{amssymb}
\usepackage{array}
\usepackage{multicol}

\newcommand{\newterm}[1]{{\alert{#1}}}
\newcommand{\figtype}[1]{{\bf \strut #1}}

\newcommand{\builtin}[1]{{\bf #1}}
\newcommand{\myvar}[1]{{\tt #1}}
\newcommand{\myfn}[1]{{\tt #1}}
\newcommand{\standin}[1]{{\it #1}}

\newcommand{\op}[1]{\, #1 \,}
\newcommand{\unify}{\op{\sqcap}}

\newcommand{\cddunify}{\stackrel{{\scriptscriptstyle <}} 
  {\sqcap}_{ca}}
\newcommand{\dunify}{\stackrel{{\scriptscriptstyle <}}
  {\sqcap}}
\newcommand{\ddunify}{\stackrel{{\scriptscriptstyle <>}}
  {\sqcap}}
\newcommand{\xdunify}{\stackrel{{\scriptscriptstyle <}} 
  {\sqcap}_{x}}
\newcommand{\csddunify}{\stackrel{{\scriptscriptstyle <}}
  {\sqcap}_{cs}}


\newcommand{\modal}[1]{\langle #1 \rangle}
%\newcommand{\hidden}[1]{}
\newcommand{\hidden}[1]{#1}
\newcommand{\notpublic}[1]{}% public version
%\newcommand{\notpublic}[1]{#1}
\newcommand{\lrulearrow}{\mapsto}

\newcommand{\lkbentryname}[1]{{\texttt{#1}}}

%\newcommand{\pred}[1]{\mbox{#1$'$}}

\newcommand{\figtext}{\normalsize}

\newcommand{\sentbound}{$\langle\mbox{s}\rangle$}
\newcommand{\sentbend}{$\langle\mbox{/s}\rangle$}
\newcommand{\affix}{\ \mbox{}\hat{\mbox{}}\mbox{}\ }

\newenvironment{pseudocode}
{\begin{tabbing}}
{\end{tabbing}}

\newcommand{\pcvar}[1]{{\color[rgb]{0,0,1}{\it #1}}}

\newcommand{\edge}[1]{[#1]}

\newcommand{\avmplus}[1]{{\setlength{\arraycolsep}{0.8mm}	
                       \renewcommand{\arraystretch}{1.2} %1.2
                       \left[ 			
                       \begin{array}{l}
                       \\[-5mm] #1 \\[-5mm] \\
                       \end{array} 		
                       \right]
                    }}
\newcommand{\attval}[2]{{\mbox{\normalsize\sc #1}\ \ {{#2}}}}
\newcommand{\attvaltyp}[2]{{\mbox{\normalsize\sc #1}\ \ \ \ {\myvaluebold{#2}}}}
\newcommand{\myvaluebold}[1]{{\mbox{\normalsize {\bf #1}}}}
\newcommand{\ind}[1]{{\setlength{\fboxsep}{0.25mm} \: \fbox{{\small #1}} \:}}


\newcommand{\savmplus}[1]{{\setlength{\arraycolsep}{0.5mm}	
                       \renewcommand{\arraystretch}{1.0} %1.2
                       \left[ 			
                       \begin{array}{l}
                       \\[-5mm] #1 \\[-5mm] \\
                       \end{array} 		
                       \right]
                    }}
\newcommand{\sattval}[2]{{\mbox{\small\sc #1}\ {{#2}}}}
\newcommand{\sattvaltyp}[2]{{\mbox{\small\sc #1}\ \ {\smyvaluebold{#2}}}}
\newcommand{\smyvaluebold}[1]{{\mbox{\small {\bf #1}}}}

% commands for figures
\newcommand{\figfeat}[1]{{\sc \strut #1}}
\newcommand{\nodedot}{\circle*{5}}
\newcommand{\feature}{\sc}
\newcommand{\type}{\bf}

\newenvironment{myeg}
{\begin{trivlist} \item \color[rgb]{0,1,0}}
{\end{trivlist}}

\newenvironment{indented}
{\begin{trivlist} \item}
{\end{trivlist}}

\newcommand{\egtext}[1]{{\color[rgb]{0,1,0} #1}}

\newcommand{\redthing}[1]{{\color[rgb]{1,0,0} #1}}

\newcommand{\bluething}[1]{{\color[rgb]{0,0,1} #1}}

\newcommand{\purplething}[1]{{\color[rgb]{1,0,1} #1}}

\newcommand{\cfga}{$\rightarrow$}

\newcommand{\argmax}{\operatornamewithlimits{argmax}}

\newcommand{\fvec}{\vec{f}}
% use in math mode

\title{Natural Language Processing and IR}
\author{Ann Copestake}
\institute{FGV/EMAp}

\newcommand{\pred}[1]{{\mbox{#1}'}}
\newcommand{\qeq}{$=_q$ }

\begin{document}

\begin{frame}{Corpora}

\begin{itemize}
\item \newterm{corpus}: text that has been collected
for some purpose.
\item \newterm{balanced corpus}: texts representing different genres\\
\newterm{genre} is a type of text (vs domain)
\item \newterm{tagged corpus}: a corpus annotated with POS tags
\item \newterm{treebank}: a corpus annotated with parse trees
\item specialist corpora --- e.g., collected to
train or evaluate particular applications
\begin{itemize}
\item Movie reviews for sentiment classification
\item Data collected from simulation of a dialogue system
\end{itemize}
\end{itemize}
\end{frame} 


\begin{frame}{Part of speech tagging}

They can fish.\\
\begin{itemize}
\item They\_pronoun can\_modal fish\_verb.\\
(`can' meaning `are able to')
\item They\_pronoun can\_verb fish\_plural-noun.\\
(`can' meaning `put into cans')
\end{itemize}
\newterm{Ambiguity}\\
{\it can}: modal verb, verb, singular noun\\
{\it fish}: verb, singular noun, plural noun
\end{frame}


\begin{frame}{Tagset}

\newterm{tagset}: standardized codes for fine-grained parts of speech.\\  
CLAWS 5: over 60 tags, including:\\[0.1in]
\begin{tabular}{ll | ll}
NN1 & singular noun & NN2 & plural noun\\
PNP & personal pronoun & VM0 & modal auxiliary verb\\
VVB & base form of verb & VVI & infinitive form of verb
\end{tabular}\\[0.1in]

\begin{itemize}
\item They\_PNP can\_VM0 fish\_VVI .\_PUN 
\item They\_PNP can\_VVB fish\_NN2 .\_PUN
\item They\_PNP can\_VM0 fish\_NN2 .\_PUN \alert{  no full parse}
\item etc
\end{itemize}

\end{frame}

\begin{frame}{Why POS tag?}

Coarse-grained syntax / word sense disambiguation:
fast, so applicable to very large corpora.
\begin{itemize}
\item Some linguistic research and lexicography: e.g.,
how often is {\it tango} used as a verb? {\it dog}?
\item Named entity recognition and similar tasks 
(finite state patterns over POS tagged data).
\item Features for machine learning e.g., sentiment classification.
(e.g., {\it stink\_V}\/ vs {\it stink\_N}\/).
\item Fast preliminary processing for full parsing: provide guesses at unknown words,
cut down search space.
\end{itemize}

\end{frame}


\begin{frame}
\frametitle{Stochastic part of speech tagging using Hidden Markov Models (HMM)}
\begin{enumerate}
\item Start with untagged text.
\item Assign all possible tags to each word in the text on the basis
of a lexicon that associates words and tags.
\item Find the most probable sequence (or n-best sequences) of tags, based on
probabilities from the training data.
\begin{itemize}
\item lexical probability: e.g., is {\it can}
most likely to be VM0, VVB, VVI or NN1? 
\item and tag sequence probabilities: e.g., 
is VM0 or NN1 more likely after PNP?
\end{itemize}
\end{enumerate}
\end{frame}



\begin{frame}
\frametitle{Assigning probabilities}

Estimate tag sequence: $n$ tags with the maximum probability, given
$n$ words:
\[
\hat{t}^n_1 = \argmax_{t^n_1} P(t^n_1 | w^n_1)
\]


By Bayes theorem:
\[
P(t^n_1 | w^n_1) = \frac{  P(w^n_1 | t^n_1) P(t^n_1)}{P(w^n_1)}
\]

but $P(w^n_1)$ is constant:

\[
\hat{t}^n_1 = \argmax_{t^n_1} P(w^n_1 | t^n_1) P(t^n_1) 
\]


\end{frame} 

\begin{frame}
\frametitle{Bigrams}

Bigram assumption: probability of a tag depends on previous 
tag, hence product of bigrams:
\[ P(t^n_1) \approx \prod^n_{i=1} P(t_i | t_{i-1}) \]

Probability of word estimated on basis of its tag alone:
\[ P(w^n_1 | t^n_1) \approx \prod^n_{i=1} P(w_i | t_i) \]

Hence:
\[
\hat{t}^n_1 =  \argmax_{t^n_1} \prod^n_{i=1} P(w_i | t_i) P(t_i | t_{i-1}) 
\]

\end{frame} 

\begin{frame}
\frametitle{Example}

Tagging: {\it they fish} (ignoring punctuation)

Assume PNP is the only tag for {\it they},
and that {\it fish} could be NN2 or VVB.

Then the estimate for PNP NN2 will be:

\[
P(\mbox{they} | \mbox{PNP})\ P(\mbox{NN2}|\mbox{PNP})\ P(\mbox{fish}|\mbox{NN2})
\]

and for PNP VVB:

\[
P(\mbox{they} | \mbox{PNP}) \ P(\mbox{VVB}|\mbox{PNP}) \ P(\mbox{fish}|\mbox{VVB})
\]


\end{frame}


\begin{frame}[fragile]
\frametitle{Training stochastic POS tagging}

\begin{semiverbatim}
They_PNP used_VVD to_TO0 can_VVI fish_NN2 in_PRP 
those_DT0 towns_NN2 ._PUN But_CJC now_AV0 few_DT0 
people_NN2 fish_VVB in_PRP these_DT0 areas_NN2 
._PUN
\end{semiverbatim}
\pause
\begin{tabular}{lll}
sequence  &     count &  bigram probability\\ \hline \hline
NN2        &      4    &           \\ \hline
NN2 PRP    &      1   &  0.25 \\
NN2 PUN    &      2   &  0.5 \\
NN2 VVB    &      1   &  0.25
\end{tabular}\\[0.1in]
\pause
Also lexicon: fish NN2 VVB
\end{frame} 

\begin{frame}
\frametitle{Assigning probabilities, more details}
\begin{itemize}
\item Maximise the overall tag sequence probability --- e.g.,
use Viterbi.
\item Actual systems use trigrams --- smoothing and backoff are critical.
\item Unseen words: these are not in the lexicon, so use all possible
\newterm{open class} tags, possibly restricted by 
morphology.
\end{itemize}
\end{frame} 

\subsection{Evaluation in general, evaluation of POS tagging}


\begin{frame}
\frametitle{Evaluation of POS tagging}

\begin{itemize}
\item percentage of correct tags
\item one tag per word (some systems give multiple tags when uncertain)
\item over 95\% for English on normal corpora
(but note punctuation is unambiguous)
\item performance plateau about 97\% on most commonly used test set for English
\item \newterm{baseline} of taking the most common tag gives 90\% accuracy
\item different tagsets give slightly different results: utility of tag
to end users vs predictive power
\end{itemize}

\end{frame} 

\begin{frame}
\frametitle{Evaluation in general}
\begin{itemize}
\item \newterm{Training data and test data}\   Test data must be
kept unseen, often 90\% training and 10\% test data.
\item \newterm{Baseline} 
\item \newterm{Ceiling}\ 
Human performance on the task, where the ceiling is the
percentage agreement found between two annotators
(\newterm{interannotator agreement})
\item \newterm{Error analysis} Error rates are nearly always 
unevenly distributed.
\item \newterm{Reproducibility} 
\end{itemize}

\end{frame} 

\begin{frame}
\frametitle{Representative corpora and data sparsity}

\begin{itemize}
\item test corpora have to be representative of the actual application
\item POS tagging and similar techniques are not 
always very robust to differences in genre
\item balanced corpora may be better, but still
don't cover all text types
\item communication aids: extreme difficulty in obtaining data, text corpora
don't give good prediction for real data
\end{itemize}
\end{frame}
 

\end{document}


%%% Local Variables:
%%% mode: latex
%%% TeX-master: t
%%% End:
